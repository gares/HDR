\documentclass[a4paper, 11pt]{book}
\usepackage[utf8]{inputenc}
\usepackage{minted}
\usepackage{graphicx}
\usepackage{hyperref}
\hypersetup{
    colorlinks,
    linkcolor={red!50!black},
    citecolor={blue!50!black},
    urlcolor={blue!80!black}
}
\usepackage[backend=biber,style=alphabetic]{biblatex}
\DeclareUnicodeCharacter{2200}{$\forall$}
\addbibresource{hdr.bib}
\addbibresource{other.bib}

\DeclareSourcemap{
  \maps[datatype=bibtex, overwrite]{
    \map{
      \perdatasource{hdr.bib}
      \step[fieldset=keywords, fieldvalue={, }, appendstrict]
      \step[fieldset=keywords, fieldvalue=me, append]
    }
    \map{
      \perdatasource{other.bib}
      \step[fieldset=keywords, fieldvalue={, }, appendstrict]
      \step[fieldset=keywords, fieldvalue=they, append]
    }
  }
}
%'theorem.py:CoqLexer -x'
\newminted[elpicode]{elpi.py:ElpiLexer -x}{fontsize=\small,autogobble,escapeinside=~~,mathescape=true,frame=leftline,framerule=0pt,framesep=1em}
\newmintinline{elpi}{fontsize=\small,autogobble,escapeinside=~~,mathescape=true,frame=leftline,framerule=0pt,framesep=1em}
\newminted[coqcode]{theorem.py:XCoqLexer -x}{fontsize=\small,autogobble,escapeinside=~~,mathescape=true,frame=leftline,framerule=0pt,framesep=1em}
\newmintinline{coq}{fontsize=\small,autogobble,escapeinside=~~,mathescape=true,frame=leftline,framerule=0pt,framesep=1em}
 
\title{title}
\begin{document}

\setcounter{tocdepth}{4}
\tableofcontents
  
 \chapter{Introduction}


\section{Prologue}

\subsection{Beyond the odd order theorem}

I had the luck to be part of this project.
After its completion the good question was what it made it possible
other than having Georges Gonthier lead the project. How to make
it possible for others to build such an impressive machine checked
proof. I've always been interested into building tools, since I believe
a tool can democratize...

3 points 1) engineering practices, not new but too often
ignored by the community; 2) formalization technique known as boolean
reflection (or small scale reflection) that carves a light weight
``classical math'' framework in Coq (EM, UIP, FUNEXT) in Coq when
constructivism allows for it; 3) ``creative'' programming of the
Coq elaborator.

We focused on 3. These programs make Coq behave as an informed readers,
that is a reader that knows some base facts and that is expected to
be able to combine them using some well known rules. It was customary,
at that time, to see Coq as a ``dumb reader'': you have to explain
every single detail to him. These programs changed the game  for the
oothm and made it possible to use notations as in math, that is convoy
a lot of information implicitly, by convention.

The objective was to help users craft these programs. Programming
the elaborator as we did was hard. Structuring the library to that
knowledge could be organized by the author an retrieved by these
programs was even harder. The programming mechanism is called CS,
and is strictly related to the UH I studied during my PhD.
Independently, about at the same time,
Coq got extended with type classes, which happened to be usd for similar
purposes, and immediately raised similar programming issues.

To my eyes, these programs were strikingly related to Prolog.
Our programs were organized in rules and driven by unification.
At that time I had barely heard about Prolog.
It happened that the Parsifal team
at Inria Saclay worked on a language or that family
and thanks to Assia Mahboubi and Stephane Legrand I could give a talk
at a STTT workshop they were hosting.

In my talk I did explain some of the programs we wrote and why they
were helping us writing math in Coq. Dale Miller, asked a few questions
and suggested that his $\lambda$Prolog language was probably what I
was looking for, but since I did not know anything about it I could
just nod and ask to continue to the discussion offline.

not just the rules but also their animation and maybe also the entire
elab algorithm.

Dale was kind
to set up a meeting with Claudio Sacerdoti and myself. 

\subsection{The beginning: a snowy day}

On that day ``Paris'' was frozen, in the sense that a supposedly
``show storm'' convinced authorities to close buildings and other facilities,
reduce public transportations to the bare minimum. For some reasons
the XXX building was not closed, so the 3 of us agreed to buy a sandwich,
walk up the hill (with boots) and meet in a desert building.

After two hours of gentle introduction, after having savoured the
elegance of \elpiinline{copy}, I clearly remember I say ``sold...'',
but I also completed the sentence with ``how do I do evars?''.

$\lambda$Prolog could easily describe the rules my programs were
made of and elegantly manipulate the data type of Coq terms that
notably contains many binders. But also holes, evars in Coq slang,
that are the other beast one has to tame in order to implement an
elaborator and, in general, the outern layers of a toll like Coq.

It was the beginning.

Teyjus in bologna, we could write a toy elaborator in one day
but we could not find a good solution. Reification is ugly, one has
to pass a subst and reimplement unif. At the same time reusing unif
variables seemed possible, but extremely hard since the pure core
of LP, and Teyjus, don't really give you systems to control their
assignment and an elaborator takes in input a term with holes
and returns a term where only some holes are assigned...

It was worth trying but using teyjus in Matita or Coq was too hard.
I write a POC interpreter of LP in OCaml for embedding it into
Matita and Coq, elpi. It was easy to add hacks and experiment with
it, and that ... Elab in matita was kind of working, we could run
it on the arithmetic library. It was clear that:
the perf were bad, very bad, like 22K times bad.
the hacks could be explained in terms of constraints and rules to
suspend, resume, combined them.


\section{Elpi}
Elpi is both a language and an implementation of that language.

With CSC we rewrote the runtime, first the FO part then the HO part
and 


\chapter{Elpi the language: \emph{de A \`a Z}}
\cite{ridoux1998lambda}

\section{Prolog}

\begin{elpicode}
pred age o:string, o:int.
pred older o:string, o:string.
\end{elpicode}
    

\section{$\lambda$Prolog}

\begin{elpicode}
kind term type.
type lam (term -> term) -> term.
type app term -> term -> term.

pred copy i:term, o:term.
copy (lam F) (lam G) :- pi x\ copy x x => copy (F x) (G x).
copy (app A B) (app C D) :- copy A C, copy B D.
\end{elpicode}

\section{Constraint Handling Rules}

\cite{chr}
\cite{10.1007/978-3-540-27775-0_7}

\begin{verbatim}
- 1 for each active constraint (just declared via declare_constraint)
 - 2 for each rule
  - 3 for each position j of the active constraint (from 0 to n)
    - 4 for each permutation of constraints having the active at j
      - 5 try apply the rule
        - 6 remove constraints to be removed from
            the current set of permutations and the set of
            active or passive constraints
\end{verbatim}

\begin{figure}
\begin{elpicode}
kind nat type.
type zero nat.
type succ nat -> nat.

pred odd i:nat.
pred even i:nat.
even zero.
odd (succ X) :- even X.
even (succ X) :- odd X.
even X :- var X, new_constraint (even X) [X].
odd X :- var X, new_constraint (odd X) [X].

pred double i:nat, o:nat.
double zero zero.
double (succ X) (succ (succ Y)) :- double X Y.
double X Y :- var X, new_constraint (double X Y) [X].

constraint even odd double {
  rule (even X) (odd X) <=> fail.
  rule (double _ X) <=> (even X).
}
\end{elpicode}
\caption[even odd]{Even and odd}
\end{figure}

\section{Elpi = $\lambda$Prolog + CHR}

constraints can be suspended goals or just data attached to variables.

in the first case,from a proof theory POV CHR is meta, it seen the unfinished AND
branches of the proof search tree and can take decisions, like merging two. a rule can
be given the meaning of $bla$. 

rules can also be non-logical, eg manage a state that is not part of 
the proof, e.g. one could fails when a rule is used more than 2 times.
sure, constraints are backtracked, CHR cannot see OR branches.

 
Constraints and CHR serve many purposes:
\begin{itemize}
  \item justify/implement some non-logical operations
  \item implement a state and its management
  \item schedule constraints
\end{itemize}

In mainstream programming one has a state. In purely functional language
one has devices to hide state passing (monads). We prefer the former.

When one uses uvar of ML to model uvars of OL he needs a state, for example
to assign a PL type to them, and means to avoid multiple type declarations for the same
uvar, or read the type.


\subsection{Syntactic sugar}

\subsubsection{Namespaces}

accumulate and no confusion.

\begin{elpicode}
namespace n1 {
  pred p.
}
q :- n1.p
shorten n1.{ p }.
q :- p.
\end{elpicode}

note that a type/mode declaration declares the belonging to a namespace.

\begin{elpicode}
p.
namespace n2 {
  pred p.
  q :- p.
}
\end{elpicode}

\subsubsection{Spilling}

names are the most important form of documentation.
LP forces one to name intermediate results, as in the ANF
in FP.

\begin{elpicode}
rapp L1 L2 L :- L = {rev {append L1 L2} }.
rapp L1 L2 L :- append L1 L2 L3, rev L3 L4, L = L4.
\end{elpicode}

\section{Example: Hindley Milner type inference}

the algorithm is meta since it needs to access the context and
in particular see the unassigned variables.

\begin{figure}
\begin{elpicode}
% terms
kind term type.
type global  string -> term.
type app term -> term -> term.
type lam (term -> term) -> term.
type let term -> ty -> (term -> term) -> term.
type eq  term -> term -> term.

% type expressions
kind tye type.
type (==>) tye -> tye -> tye.  

% types
kind ty type.
type all    eq? -> (tye -> ty) -> ty.
type mono   tye -> ty.

% type quantification
kind eq? type.
type any eq?. % any type
type eqt eq?. % type with an equality test

% builtin types
type int   tye.
type bool  tye.
type list  tye -> tye.
type pair  tye -> tye -> tye.
\end{elpicode}
\caption[syntax]{Syntax of terms and types\label{hm:syntax}}
\end{figure}


\begin{figure}
\begin{elpicode}
pred of i:term, o:ty.
of (global "1")      (mono int).
of (global "2")      (mono int).
of (global "3")      (mono int).
of (global "plus")   (mono (int ==> int ==> int)).
of (global "[]")    (all any x\ mono (list x)).
of (global "::")    (all any x\ mono (x ==> list x ==> list x)).
of (global "size")  (all any x\ mono (list x ==> int)).
of (global "undup") (all eqt x\ mono (list x ==> list x)).
of (global ",")     (all any x\ all any y\ mono (x ==> y ==> (pair x y))).
\end{elpicode}
\caption[type assignments]{Type assignments\label{hm:env}}
\end{figure}
  
\begin{figure}
\begin{elpicode}
pred specialize i:ty, o:tye.
specialize (mono T) T.
specialize (all any F) T :- specialize (F Fresh_) T.
specialize (all eqt F) T :- specialize (F Fresh) T, eqbar Fresh.

pred eqbar i:tye.
eqbar bool.
eqbar int.
eqbar (list A) :- eqbar A.
eqbar (pair A B) :- eqbar A, eqbar B.

eqbar T :- var T, new_constraint (eqbar T) [T,_].
eqbar T :- print "KO: type" T "has no equality", halt.
\end{elpicode}
\caption[schema elimination]{Type schema elimination\label{hm:elim}}
\end{figure}

\begin{figure}
\begin{elpicode}
% theta carries the list of type variables for which eqbar
% has to hold
pred theta i:list tye.
theta L :- new_constraint (theta L) [_].

% gammabar is not a real constraint, but rather a query to the meta
% level to compute a polymorphic type out of a monomorphic one and
% its context
pred gammabar i:ty, o:ty.
gammabar (mono T) TS :- new_constraint (gammabar (mono T) TS) [_].

% constraint store %
constraint of gammabar eqbar theta {
  rule (theta L)                    % matched
        \  (G ?- gammabar T TS)     % matched and removed
        |  (generalize L G T POLYT) % guard + syntesis
      <=> (TS = POLYT).             % new goal

  rule (eqbar V) \ (theta L) | (not(mem L V)) <=> (theta [V | L]).
}

pred generalize i:list tye, i:list prop, i:ty, o:ty.
generalize Theta Gamma (mono T) PolyT :-
  free-ty (mono T) [] VT,
  free-gamma Gamma [] VGamma,
  filter VT (x\ not(mem VGamma x)) ToQuantify,
  bind ToQuantify Theta T PolyT.

pred bind i:list tye, i:list tye, i:tye, o:ty.
bind [] _ T (mono T1) :- copy T T1.
bind [X|XS] Theta T (all E x\ T1 x) :-
  if (mem Theta X) (E = eqt) (E = any),
  pi c\ copy X c => bind XS Theta T (T1 c).

pred free-ty i:ty, i:list tye, o:list tye.
free-ty (mono X) L L1 :- free X L L1.
free-ty (all _ F) L L1 :- pi x\ free-ty (F x) L L1.

pred free-gamma i:list prop, i:list tye, o:list tye.
free-gamma [] L L.
free-gamma [of _ T|X] L L2 :- free-ty T L L1, free-gamma X L1 L2.

pred free i:tye, i:list tye, o:list tye.
free int L L.
free bool L L.
free (list A) L L1 :- free A L L1.
free (pair A B) L L2 :- free A L L1, free B L1 L2.
free (A ==> B) L L2 :- free A L L1, free B L1 L2.
free (uvar _ _ as X) L L1 :- if (mem L X) (L1 = L) (L1 = [X|L]).
\end{elpicode}
\caption[schema introduction]{Type schema introduction\label{hm:intro}}
\end{figure}


\begin{figure}
\begin{elpicode}
of (app H A) (mono T) :-
  of H (mono (S ==> T)),
  of A (mono S).

of (lam F) (mono (S ==> T)) :-
  pi x\ of x (mono S) => of (F x) (mono T).

of (let E PT B) (mono TB) :-
  of E (mono T),
  gammabar (mono T) PT,
  pi x\ of x PT => of (B x) (mono TB).

of (eq LHS RHS) (mono bool) :-
  of LHS (mono T),
  of RHS (mono T),
  eqbar T.

of X (mono T) :- of X (all E Poly), specialize (all E Poly) T.
\end{elpicode}
\caption[bidirectional]{Bidirectional typing\label{hm:bidir}}
\end{figure}

\begin{figure}
\begin{elpicode}
of (app H A) (mono T) :-
  of H (mono TH),
  of A (mono TA),
  assert H TH (TA ==> T).

of (lam F) (mono (S ==> T)) :-
  pi x\ of x (mono S1) => of (F x) (mono T1),
  assert (lam F) (S1 ==> T1) (S ==> T).

of (let E PT B) (mono TB) :-
  of E (mono T),
  gammabar (mono T) PT,
  pi x\ of x PT => of (B x) (mono TB1),
  assert (B x) TB1 TB.

of (eq LHS RHS) (mono B) :-
  of LHS (mono TL),
  of RHS (mono TR),
  eqbar TL,
  assert RHS TR TL,
  assert (eq LHS RHS) bool B.

of X (mono T) :- of X (all E Poly), specialize (all E Poly) T1, assert X T1 T.

pred assert i:term, i:tye, i:tye.
assert _ TY ETY :- TY = ETY, !.
assert T TY ETY :-
  print "KO: term" T "has type" TY "but its context expects" ETY, halt.
\end{elpicode}
\caption[monodirectional]{Monodirectional typing\label{hm:mono}}
\end{figure}

\section{Homoiconicity and rule based languages}

\section{Pitfalls}

\begin{enumerate}
  \item precedence of \elpiinline{=>}
  \item \elpiinline{sigma} considered harmful
  \item restriction should be fatal
\end{enumerate}

\chapter{Elpi the software: \emph{de A \`a Z}}

\section{The prototype}

\section{The second runtime}

\cite{dunchev15lpar}

\subsection{The $L_{\lambda}$ fragment}
\subsection{The $L_{\lambda}^{\beta}$ fragment}

\cite{Michaylov1993HigherOrderLP}

\subsection{Indexing}
\subsubsection{Patricial tree}
\subsubsection{Discrimination tree}
\subsubsection{Hash map}

\section{The API}

\subsection{Quotations}
\subsection{FFI}
\section{The tools}

\subsubsection{Trace browser}

\chapter{Coq-Elpi}

While Elpi is a standalone language and software project, it was designed to
play the role of an \emph{extension language} for Coq. The glue between Elpi and
Coq is called Coq-Elpi.

My definition of this role, \emph{extension language}, is largely influenced
by the Lua programming language~\cite{10.5555/1200583}. Lua is an extension
language for applications written in C and is widely used in the Opens Source
world and in the gaming industry.
Its purpose is to provide an easy way to extend the host application.
It is easy to host Lua since its FFI is well curated, and thanks to that
exposing the internals of the application to the extension language requires a
limited effort to the application developer. It is easier to program in Lua,
rather than C, because the language is higher level, e.g. is features automatic
memory management and provides dictionaries as a builtin data structure.
Finally, it is easy for a user to get started with Lua because he does not need
to set up a proper development environment, the host application is sufficient
since Lua is an interpreter.\\
Elpi tries to do the same for OCaml, and Coq is the host application of interest.

Even if I'm not a big fan of it, in academia the same role is often called
meta-programming framework. Coq's main data type, terms, are programs and Elpi
programs do manipulate Coq programs. In this sense Elpi programs live at the
meta level.

\section{Why extending Coq in OCaml is hard}

Luckily OCaml features automatic memory management, types and algebraic data.
It is much, much higher level than C. Why it is hard to extend Coq then?

The first difficulty is that the complexity of the main Coq data type, terms,
is not completely hidden by the algebraic data types provided by the programming
language. The missing features are encoded and it is hard to completely hide
it to the programer (via curated APIs). In particular Coq terms feature binders
and holes. 

Binders pose two problems of their own and interact badly with holes.
The first problem is that they are typically encoded with numbers, De Bruijn
indexes, and it is just too easy to forget to shift a term.  The second program
is that when a binder is crossed on must always remember something about it,
typically the type of the bound variable. Hence the programs have to pass around
typing contexts (aka environment). It is tempting to not do it upfront, but
then one has to be disciplined. find mantra mc bride.

Holes are missing subterms and they come with metadata, a sequent.
Since the same hole can occur non linearly a single sequent is stored.
Moreover since binders may be reduced away each occurrence has an explicit
substitution. So there is a state that accompanies the terms, a state to be
threaded in a functional setting. This state also gathers univ constraints.
Assignment also part of this state, see also ``evar sensitive''. This is
a reified heap, we are back at memory management, pointer dereferencing (with no
types), no gc. The only thing that is easy is backracking, since the map is
purely functional.

The second difficulty is no inherent in OCaml, but rather an artifact of the
history of Coq. The API are not curated. Exposing the API in a manual, althought
not very demanding, way is key to rationalize them.


Last setting up the dev environment. I believe that can be eased with doc, but
there is another advantage in a interpreter, that you can iterate faster. Eg
in OCaml loading code is possible, but not unloading. This is even better in
a rule based language where far from the place where a code is writtne one
can add a rule to mask/improve/debug a piece of code.

\section{Vernacular language integration}

While the language of Coq terms, gallina, is central to Coq-Elpi, there is another
language we have to describe first, the vernacular. That ``outern'' language
lets one organize the formalized knowledge, in particular give names to terms
and attach to them meta data.

Coq-Elpi extends the vernacular language with a few commands to declare, run
and modify Elpi programs.



\begin{figure}
\begin{coqcode}
Elpi Command hello.
Elpi Accumulate lp:{{
  main [str X] :- coq.say "Hello" X.
}}.
Elpi hello "reader".

Fail Elpi hello 46.

Elpi Accumulate lp:{{
  main [int X] :- coq.say "Hello" X.
}}.

Elpi hello 46.
\end{coqcode}
\caption[Vernacular]{Vernacular\label{vernac}}
\end{figure}

\subsection{Database and homoiconicity}

share data between programs, and programmatically add data.

\begin{figure}
  \begin{coqcode}


    Elpi Db age.db lp:{{

    % A typical Db is made of one main predicate
    pred age o:string, o:int.
  
    % the Db is empty for now, we put a rule giving a
    % descriptive error and we name that rule "age.fail".
    :name "age.fail"
    age Name _ :- coq.error "I don't know who" Name "is!".
  
  }}.
  Elpi Command age. 
  Elpi Accumulate Db age.db.  (* we accumulate the Db *)
  Elpi Accumulate lp:{{
  
    main [str Name] :-
      age Name A,
      coq.say Name "is" A "years old".
  
  }}.
  Elpi Typecheck. 
  
  Elpi age bob.
    
  Elpi Accumulate age.db lp:{{
  
    :before "age.fail"     % we place this rule before the catch all
    age "bob" 24.
  
  }}.
  
  Elpi age bob.
  
  
  Elpi Command set_age.
  Elpi Accumulate Db age.db.
  Elpi Accumulate lp:{{
    main [str Name, int Age] :-
      TheNewRule = age Name Age,
      coq.elpi.accumulate _ "age.db"
        (clause _ (before "age.fail") TheNewRule).

  }}.
  
  Elpi set_age "alice" 21. 
  Elpi age "alice".
  \end{coqcode}
  \caption[Databases]{Databases\label{databases}}
  \end{figure}

  
\section{HOAS of terms and contexts}

\begin{figure}
\begin{elpicode}
% Global objects: inductive types, inductive constructors, definitions
kind gref type.
type const constant -> gref. % Nat.add, List.append, ...
type indt inductive -> gref. % nat, list, ...
type indc constructor -> gref. % O, S, nil, cons, ...

kind term type.

type sort  sort -> term. % Prop, Type@{i}

% constants: inductive types, inductive constructors, definitions
type global gref -> term.
type pglobal gref -> univ-instance -> term.

% binders: to form functions, arities and local definitions
type fun  name -> term -> (term -> term) -> term.         % fun x : t =>
type prod name -> term -> (term -> term) -> term.         % forall x : t,
type let  name -> term -> term -> (term -> term) -> term. % let x : T := v in

% other term formers: function application, pattern matching and recursion
type app   list term -> term.                   % app [hd|args]
type match term -> term -> list term -> term.   % match t p [branch])
type fix   name -> int -> term -> (term -> term) -> term. % fix name rno ty bo

type primitive primitive-value -> term.
\end{elpicode}
\caption[terms]{Terms\label{hoas:term}}
\end{figure}
  
\begin{figure}
\begin{elpicode}
pred decl i:term, o:name, o:term. % Var Name Ty
pred def  i:term, o:name, o:term, o:term. % Var Name Ty Bo
\end{elpicode}
\caption[context]{Context\label{hoas:context}}
\end{figure}

\section{HOAS of holes (missing terms)}

holes come with data expressing an invariant
invariant checked when the hole materializes


\begin{figure}
\begin{elpicode}
pred evar i:term, i:term, o:term. % Evar Ty RefinedSolution
evar X Ty R :- not(var R), !, coq.typecheck R Ty ok, X = R.
 
constraint declare-evar evar def decl {

% Override the actual context
rule \ (declare-evar Ctx RawEv Ty Ev) <=> (Ctx => evar RawEv Ty Ev).
   
}
\end{elpicode}
\caption[holes]{Holes\label{hoas:holes}}
\end{figure}

\chapter{Coq applications}

In this chapter we survey the main applications developed on top of Coq-Elpi.

\section{Derive}

The ``derive'' application is a framework to register automatic
code generation, typically in response to the declaration of a new inductive
data type.

Coq itself is know to generate induction principles and equality tests.
Unfortunately it is also known to generate bad induction principles when
containers (e.g. lists) are involved and often fail to generate useful
equality tests.

I saw the opportunity to improve on that starting with the L3 internship of
Luc Chabassier that wrote the first prototype in the summer of 2017.
He succeeded in generating
equality tests and their proofs in two months. While most of the credit goes to
his learning skills, the experience confirmed that Coq-Elpi was to support
research in this domain. Indeed the solution implemented by Luc, although
correct, was not very satisfactory since it was not very modular. Improving
on that lead to \cite{tassi:hal-01897468}, but before that I need to introduce
the swiss army knife of boilerplate code generation: parametricity.

\subsection{Parametricity}

The so called parametricy translation \cite{keller_et_al:LIPIcs.CSL.2012.381}
is a family of procedures that given \emph{any} Coq term $t$ of type $r$
give both a predicate $R$ and a proof $T$ that $R$ holds for $t$, that is $T : R~t$
in the unary case. For example the unary parametricity translation of
\coqinline{nat : U} is \coqinline{is_nat : nat → U} and
the translation of \coqinline{3} if a proof that \coqinline{is_nat 3} (we write
\coqinline{U} for the universe, e.g. \coqinline{Prop}).

The translation becomes interesting when the type has parameters such as
\coqinline{A} in \coqinline{list A}. In that case the translation builds
terms and type parametric in \coqinline{A} and in its translation. For example
the translation of \coqinline{list A} is a predicate
\coqinline{is_list : ∀A, (A → U) → (list A → U)},
and the translation of \coqinline{cons : ∀A, A → list A → list A} is
\begin{coqcode}
is_cons : ∀A (isA : A → U),
  ∀(a : A), isA a →
  ∀l, is_list A isA l →
    is_list A isA l
\end{coqcode}

The unary parametric of a container type is a predicate that
asserts that each term contained has a given property, \coqinline{isA}
in the example above. The unary parametricity translation is key to express in a
systematic way that a property holds deep inside a container.

The parametricity translation is meta, it cannot be implemented in Gallina
itself hence one needs a meta language to implement it. Its implementation
in Coq-Elpi was done by Cohen Cyril in the fall of 2017, as he was interested
in exploiting the binary version of the translation to transfer properties
automatically, possibly in the context of the Coq-EAL project.


\subsection{Deep induction}

Deep induction principles are said ``deep'' in contrast to the ones generated
by Coq that, in the presence of containers, are shallow (and proof theoretically
weak). As an example we can take the data type of rose trees.

\begin{coqcode}
Inductive rtree A : U :=
| Leaf (a : A)
| Node (l : list (rtree A)).
\end{coqcode}

The induction principle generated by Coq is

\begin{coqcode}
Lemma rtree_ind : ∀A (P : rtree A → U),
  (∀a : A, P (Leaf A a)) →
  (∀l : list (rtree A), P (Node A l)) →
  ∀t : rtree A, P t.
\end{coqcode}

That is weak since no element of \coqinline{l} in \coqinline{Node A l}
verifies \coqinline{P}. In \cite{tassi:hal-01897468} I study the synthesis
for a stronger principle where \coqinline{P} holds ``deep'' inside \coqinline{l}.

\begin{coqcode}
Lemma rtree_induction A is_A (P : rtree A → U) :
  (∀a, is_A a → P (Leaf A a)) →
  (∀l, is_list (rtree A) P l → P (Node A l)) →
     ∀t, is_rtree A is_A t → P t.
\end{coqcode}

There are two characteristics of this principle worth analyzing.

First and foremost it is deep since the assumption
\coqinline{is_list (rtree A) P l} gives access to \coqinline{P} on all
elements of \coqinline{l}.
In order to access \coqinline{P} one can combine lemmas such as
\begin{coqcode}
Lemma is_list_funct A P Q : (∀ a, P a → Q a) → ∀ l, is_list A P l → is_list A Q l.
\end{coqcode}

with the induction principle of the container
\begin{coqcode}
Lemma list_induction A (is_A: A → U) (P: list A → U):
  P nil →
  (∀ a (pa : is_A a) l, P l → P (a :: l)) →
  ∀ l, is_list A is_A l → P l.
\end{coqcode}

Note how the former works under \coqinline{is_list} while the latter eliminates it.

\begin{coqcode}
Definition rtree_induction (A : U) (PA : A → U) (P : rtree A → U)
  (His_Leaf : ∀a : A, PA a → P (Leaf A a))
  (His_Node : ∀l : list (rtree A), is_list (rtree A) P l → P (Node A l)) :=
  fix IH (s1 : rtree A) (x : is_rtree A PA s1) {struct x} : P s1 :=
  match x in (is_rtree _ _ s2) return (P s2) with
  | is_Leaf _ _ a Pa => His_Leaf a Pa
  | is_Node _ _ l Pl => His_Node l (is_list_functor (rtree A) (is_rtree A PA) P IH l Pl)
  end.
\end{coqcode}


In a way the induction principle states that ``if a term is of type t then it validates P''


\subsection{Natural equality tests}

Once one has induction principles he can prove properties about recursive programs.
The first is equality test.

\begin{coqcode}
Definition rtree_eq A (A_eq : A → A → bool) :=
  fix rec (t1 t2 : rtree A) {struct t1} : bool :=
  match t1, t2 with
  | Leaf a, Leaf b => A_eq a b
  | Node l, Node s => list_eq (rtree A) rec l s
  | _, _ => false
  end.
\end{coqcode}



\cite{tassi:hal-01897468}

\subsection{Fast equality tests}

This is OK but terms are quadratic

\cite{gregoire:hal-03800154}


\section{Hierarchy Builder}

The math comp library has many good points and a few bad ones. Among these
the most endangering ons is that the barrier to entry is high. Documentation
a in \cite{assia_mahboubi_2022_7118596} helps, but any motivated learner
can only read so much. In particular the last chapters of the book above
quickly grow in complexity and technicality. The problem is that, prior to
HB, defining a hierarchy of algebraics structure à la mathcomp was terribly
hard. So hard that I myself would have not get it right at the first try.

In Spring 2019 Cohen Cyril came back from a Dagstuhl seminar with ``a language for
describing the math-comp hierarchy''. He needed a way to implement his language and
Coq-Elpi had almost all the features he needed: APIs to declare records and
canonical structure instances.

To be honest, at that time Coq-Elpi lacked so many features, but the objective
at stake was both a strong motivation for adding them and also a good test
bench for their implementation.



\cite{cohen_et_al:LIPIcs.FSCD.2020.34}

\subsection{Mathematical Components 2.0}

Porting Mc to HB was a huge effort. we did it with a sprint.

\cite{affeldt:hal-03463762}

\includegraphics[width=\textwidth]{hb\_intf.png}

one should remark that not only MC2 has the same interfaces, but
also that the trend changed, much more are defined.

MCa has 64 + 24 legacy, as of 1.2.0-8-g1baa0a8


% mathcomp-1.6.1 mathcomp-1.7.0 mathcomp-1.8.0 mathcomp-1.9.0 mathcomp-1.10.0 mathcomp-1.11.0 mathcomp-1.12.0 mathcomp-1.13.0 mathcomp-1.14.0 mathcomp-1.15.0 mathcomp-1.16.0 mathcomp-1.17.0 mathcomp-1.18.0 mathcomp-1.19.0 mathcomp-2.0.0 mathcomp-2.1.0 mathcomp-2.2.0

% for t in mathcomp-1.6.1 mathcomp-1.7.0 mathcomp-1.8.0 mathcomp-1.9.0 mathcomp-1.10.0 mathcomp-1.11.0 mathcomp-1.12.0 mathcomp-1.13.0 mathcomp-1.14.0 mathcomp-1.15.0 mathcomp-1.16.0 mathcomp-1.17.0 mathcomp-1.18.0 mathcomp-1.19.0 mathcomp-2.0.0 mathcomp-2.1.0 mathcomp-2.2.0 `git describe --tags`; do printf "$t,%d,%d\n" "`git grep ^Structure $t|wc -l`" "`git grep HB.structure $t|wc -l`"; done
% for t in mathcomp-1.6.1 mathcomp-1.7.0 mathcomp-1.8.0 mathcomp-1.9.0 mathcomp-1.10.0 mathcomp-1.11.0 mathcomp-1.12.0 mathcomp-1.13.0 mathcomp-1.14.0 mathcomp-1.15.0 mathcomp-1.16.0 mathcomp-1.17.0 mathcomp-1.18.0 mathcomp-1.19.0 mathcomp-2.0.0 mathcomp-2.1.0 mathcomp-2.2.0 `git describe --tags`; do printf "$t,%s,%d,%d\n" "`git log -1 --format=%as $t`" "`git grep ^Structure $t|wc -l`" "`git grep HB.structure $t|wc -l`"; done

\section{Other applications / uses}

Coq-Elpi found applications in other projects I did not participate
into directly. Here we mention the ones we are aware of.

\subsection{Algebra Tactics}

For a long time calling ring in MC was hard, one diff
being because of term
repr and cs proj. KS wrote a kit of tactics

\cite{sakaguchi:LIPIcs.ITP.2022.29}

\begin{elpicode}
ring C {{ @GRing.opp lp:U lp:In1 }} {{ @ROpp lp:R lp:OutM1 }} Out VM :-
  coq.unify-eq { rmorphism->zmod C } U ok,
  rmorphism->ring C R, !,
  ring C In1 OutM1 Out1 VM, !,
  build.opp Out1 Out.  
\end{elpicode}

key is call to unify

\subsection{Tackt and TRocq: proof transfer tool}

The need for proof transfer, Enzo Assia and later Cyril.

\cite{DBLP:conf/cpp/Blot0CPKMV23}
\cite{10.1007/978-3-031-57262-3_10}

the thesis compaares how rules can be transposed naturally.
CHR to store state, although the wekness of debugging
facilities limited its use to that, and not data manipulation.

\subsection{EIris: ???}

https://github.com/lukovdm/MasterThesisIrisElpi/blob/e7c9f2a835a523fa8937b24f06bac061e5f50ab9/latex/thesis/thesis.pdf

\subsection{BedRock's BRICK}
\subsubsection{N.E.S. -- Namespace Emulation System}



\subsubsection{Derive plugins}

\chapter{Conclusions}
\section{Future work}
\nocite{*}
\printbibliography[title={Our Bibliography}, keyword=me]
\printbibliography[title={Bibliography}, keyword=they]
\end{document}